\documentclass[10pt]{article}
\usepackage[a4paper, left=0.4in, right=0.4in, top=0.25in, bottom=0.15in]{geometry}
\usepackage[utf8]{inputenc}
\usepackage[T1]{fontenc}
\usepackage[english]{babel}
\usepackage{fontawesome}
\hyphenation{Some-long-word}

\usepackage{resume}

\begin{document}

\begin{center}
    {\Huge HIMANSHU \textbf{MUNDHRA}}\\[0.5ex]
\end{center}
% \begin{tabular*}{\textwidth}{@{}l @{\extracolsep{\fill}} r@{}}
%   \begin{tabular}{@{}l@{}} 
%      AE-312, MS Hall of Residence,\\ 
%      \href{http://www.iitkgp.ac.in}{Indian Institute Of Technology, Kharagpur}\\
%      West Bengal, India - 721302
%   \end{tabular} & 
%   \begin{tabular}{@{}r@{}} 
%     \faMobile\hspace{0.2ex} \textbf{+91 91639 95974} \\  
%     \Letter\hspace{0.2ex}
%     \href{mailto:himanshumundhra98@gmail.com}{himanshumundhra98@gmail.com} \\
%     \faLinkedinSquare\hspace{0.2ex}  \href{https://www.linkedin.com/in/shmundhra/}
%     {\tt linkedin.com/in/shmundhra}  \\
%     \faGithub\hspace{0.2ex} 
%     \href{https://github.com/shmundhra/}{\tt github.com/shmundhra}  \\
%     % \faMedium\hspace{0.2ex} \href{https://medium.com/@himanshumundhra98}
%     % {\tt medium.com/@himanshumundhra98}  \\
%   \end{tabular}
% \end{tabular*}
\vspace{-4.1ex}
\begin{center}
    {\small AE-312, MS Hall of Residence, IIT Kharagpur, West Bengal, India - 721 302 | \faMobile\hspace{0.2ex} \textbf{+91 91639 95974} }
\end{center}
\vspace{-4.1ex}
\begin{center}
    \Letter\hspace{0.2ex}
    \href{mailto:himanshumundhra98@gmail.com}{\small himanshumundhra98@gmail.com} | 
    \faLinkedinSquare\hspace{0.2ex}
    \href{https://www.linkedin.com/in/shmundhra/}{\small shmundhra} | 
    \faGithub\hspace{0.2ex}
    \href{https://github.com/shmundhra/}{\small shmundhra} | 
    \faMedium\hspace{0.2ex}
    \href{https://medium.com/@himanshumundhra98}{\small himanshumundhra98}
\end{center}

\vspace{-3.0ex}
\roottitle{\Large{EDUCATION}}
\spacedhrule{0.2ex}{2.0ex}
\renewcommand{\arraystretch}{1.5}
\indent \begin{tabular}{ |@{\hskip 0.125in}l @{\hskip 0.125in} |@{\hskip 0.125in}l @{\hskip 0.125in} |@{\hskip 0.125in}l @{\hskip 0.125in} |@{\hskip 0.125in}l @{\hskip 0.120in} |l }
\hline
\textbf{Degree} & \textbf{Institute / Board} & \textbf{Year} & \textbf{CGPA / \%} \\
\hline
B. Tech in Computer Science and Engineering & IIT Kharagpur  & 2016 - 2020 (Expected) & \textbf{9.57 / 10} \href{https://github.com/shmundhra/Credentials/tree/master/Academics} {\hspace{1.0ex}\faMousePointer}\\
\hline
All India Senior School Certificate Examination & Birla High School - CBSE & March, 2016 & \textbf{95.2 \%} \href{https://github.com/shmundhra/Credentials/tree/master/Academics} {\hspace{2.5ex}\faMousePointer}\\
\hline
All India Secondary School Examination & Birla High School - CBSE & March, 2014 & \textbf{10 / 10 } \href{https://github.com/shmundhra/Credentials/tree/master/Academics} {\hspace{2.3ex}\faMousePointer}\\
\hline
\end{tabular}
\\

\vspace{-1.5ex}
\roottitle{\Large{INTERNSHIP}}
\spacedhrule{0.1ex}{1.0ex}
\large {\textbf{Member of Technical Staff Intern at Rubrik, Inc.}} \normalsize
\href{https://github.com/shmundhra/Credentials/tree/master/Internships} {\hspace{0.5ex}\faMousePointer}
{\hfill} \textbf{Summer'19}\\[-1.8em]
\begin{itemize}
\item Designed a High Throughput Streaming Library for data transfer over a \textbf{Secured TCP Connection} as an API to user.\\[-2em]
\item Implemented a \textbf{duplex server-client stream} where each end can be a rewindable/non-rewindable   producer/consumer.\\[-2em]
\item \textbf{Designed} the response, request and error packet \textbf{headers} and the \textbf{protocol} to be followed from scratch.\\[-2em]
\item Implemented \textbf{multiple classes in a hierarchical structure} and developed different utilities and wrapper modules.\\[-2em]
\item Used \textbf{thread-per-connection concurrency control} mechanism and maintained synchronisation between threads using \textbf{mutex lock wrappers} to avoid the producer-consumer and stream registration race conditions.\\[-2em]
\end{itemize}

% \iffalse
\roottitle{\Large{TEACHING EXPERIENCE}}
\spacedhrule{0.1ex}{1.0ex}
\large { \textbf{Teaching Assistant} for Algorithms-I CS20003, IIT Kharagpur} \normalsize
\href{https://github.com/shmundhra/Credentials/tree/master/Teaching\%20Ventures} {\hspace{0.5ex}\faMousePointer}
{\hfill} \textbf{Jul'19 - Present}\\[-1.95em]
\begin{itemize}
    \item Organise \textbf{tutorials} for the students, set practice problems and solve with the class along with doubt clearing.\\[-1.5em]
\end{itemize}
\large { \textbf{Lecturer} at Competitive Programming Workshop, IIT Kharagpur} \normalsize
\href{https://github.com/shmundhra/Credentials/tree/master/Teaching\%20Ventures} {\hspace{0.5ex}\faMousePointer}
{\hfill} \textbf{Jan'19 - Apr'19}\\[-1.95em]
\begin{itemize}
    \item Organised and Lectured a \textbf{Competitive Programming Workshop} for the students of IIT Kharagpur.\\[-2em]
\end{itemize}
% \fi

\roottitle{\Large{PROGRAMMING REPOSITORIES}}
\spacedhrule{0.1ex}{1.0ex}
\large {\textbf{Systems Programming}} \normalsize 
\href{https://github.com/shmundhra/Systems-Programming}{\faGithub}
{\hfill} \textbf{Spring'19 - Ongoing}\\[-1.8em]
\begin{itemize}
\item Implemented a rudimentary \textbf{Command-Line Interpreter} for Linux on C++ using \textbf{forks and pipes}. \\[-1.9em]
\item Developed a \textbf{Process/Thread Scheduler} which implements the common Process Scheduling Algorithms.\\[-1.9em]
\item Simulated a \textbf{Multi-threaded mutually exclusive deadlock free} Producer-Consumer problem implementation.\\[-1.75em]
\item  
\begingroup
    \fontsize{11pt}{12pt}\selectfont
    Simulated \textbf{Virtual Memory using Demand Paging} \\[-1.75em]
\endgroup
    \begin{itemize}
    \item Created \textbf{different modules} such as Master, Scheduler, Processes and the Memory Management Unit (MMU).\\[-1.60em]
    \item Implemented message passing between modules through     \textbf{Blocking Synchronous IPC Message Queues}.\\[-1.60em]
    \item Accessed \textbf{Shared Memory} synchronously using \textbf{signals \& messages} to indicate safe and mutually exclusive access.\\[-1.60em]
    \item \textbf{Proportional allocation} of frames along with an \textbf{LRU strategy} for the Page Table and the TLB was adopted.\\[-1.70em]
    \end{itemize}
\item 
\begingroup
    \fontsize{11pt}{12pt}\selectfont
    Implemented a \textbf{Memory-Resident Unix-Like File System} \\[-1.75em]
\endgroup
    \begin{itemize}
    \item \textbf{Created a File System} which supports all Linux-type file commands like - open(), read(), write() and more.\\[-1.60em]
    \item \textbf{Linked List Implementation} where the \textbf{Free Blocks are a Bit Vector} and \textbf{Data Blocks are maintained in a FAT}.\\[-1.60em]
    \item \textbf{iNode Implementation} where the \textbf{Free Blocks are a Linked List}  and the \textbf{File Blocks are maintained in iNodes}.\\[-1.25em]
    \end{itemize}
\end{itemize}
\large {\textbf{Socket Programming}} \normalsize 
\href{https://github.com/shmundhra/Socket-Programming}{\faGithub}
{\hfill} \textbf{Spring'19 - Ongoing}\\[-1.7em]
\begin{itemize}
\item 
\begingroup
    \fontsize{11pt}{12pt}\selectfont
    Developed an \textbf{iterative FTP Server and FTP Client} following a subset of the File Transfer Protocols \\[-1.7em]   
\endgroup
    \begin{itemize}
    \item \textbf{Client sends the port} at which it is waiting for data to the Server through the \textbf{Command Channel}. \\[-1.65em]
    \item \textbf{Server initiates the connection} and a put() or get() of binary files can take place over the \textbf{Data Channel}. \\[-1.65em]
    \end{itemize}
\item 
\begingroup
    \fontsize{11pt}{12pt}\selectfont
    Developed a \textbf{Reliable User Datagram Protocol} using \textbf{MyReliableProtocol(MRP) Sockets} \\[-1.75em]
\endgroup
    \begin{itemize}
    \item Created a \textbf{Static Library} with all required functions for our protocol - socket(), send(), recv(), close() .\\[-1.60em]
    \item Created a \textbf{Concurrent Thread} which managed the Receiving of Messages and placed them into the Receive Buffer.\\[-1.60em]
    \item This Thread also managed the \textbf{Acknowledgements and the Re-transmissions} to ensure reliability. \\[-1.75em]
    \end{itemize}
\item Developed a version of the Linux-Command \textbf{\$traceroute} using \textbf{Raw Sockets} and the \textbf{TTL Field} in the IP Header.
\end{itemize}
\large {\textbf{Machine Learning}} \normalsize 
\href{https://github.com/shmundhra/Machine-Learning}{\faGithub}
{\hfill} \textbf{Spring'19 - Ongoing}\\[-1.8em]
\begin{itemize}
\item Developed a \textbf{Regression Tool using Gradient Descent} to study the effect of variation of hyperparameters on the plots.\\[-2em]
\item Developed a \textbf{Decision Tree Classifier} to classify a large dataset of news articles into alt.atheism or comp.graphics.\\[-2em]
\item Implemented a \textbf{Hierarchical Clustering} to cluster submissions to conferences by their High-Level Domains.\\[-2em]
\item Implemented a \textbf{Multi-Layer Artificial Neural Network} for Email SPAM/HAM Classification. \\[-1em]
\end{itemize}
\large {\textbf{Competitive Programming}} \normalsize 
\href{https://github.com/shmundhra/Competitive-Codes}{\faGithub}
{\hfill} \textbf{Winter'17 - Ongoing}\\[-1.8em]
\begin{itemize}
\item Contains all the codes from different platforms like CodeChef, Codeforces, InterviewBit and from my participations in competitions like ACM-ICPC, Facebook Hacker Cup, Google Code Jam and Google Kickstart. \\[-2em]
\end{itemize}

\vspace{-2.0ex}
\roottitle{\Large{PROJECTS}}
\spacedhrule{0.1ex}{1.0ex}
\large {\textbf{Multi Target Stance Detection using Graph Convolution Networks}} \normalsize
\href{}{}
{\hfill} \textbf{Ongoing}\\[-1.8em]
\begin{itemize}
    \item Aim to \textbf{assign a stance to textual data} by a user catering to a particular target or a set of related targets using GCNs.\\[-2em]
    \item \textbf{TextGCNs generate a multi-layer graph} that will incorporate the user history and comments on the topic of interest.\\[-2em]
    \item Aim to utilise \textbf{user background information} to be able to predict their stance in a more accurate manner.\\[-2em]
    \item GCNs \textbf{Semi-supervised framework} allows us to train a small dataset and achieve results similar to full-supervision.\\[-1em]
\end{itemize}
\large {\textbf{Auditorium and Room Booking System (HOVA)}} \normalsize  \href{https://github.com/shmundhra/HOVA} {\hspace{0.5ex}\faMousePointer}
{\hfill} \textbf{Spring'19}\\[-1.8em]
\begin{itemize}
    \item Developed a \textbf{Web Application} on Java NetBeans using \textbf{JSP and MySQL} to automate room booking in IIT Kharagpur.\\[-2em]
    \item Incorporated the \textbf{hierarchical structure} of room booking by including verification from the Department, Authority, Security and the AV Cell, in that order.\\[-2em]
    \item Included features of \textbf{submitting/cancelling booking requests} and seeing pending applications from the Applicant Side.\\[-2em]
    \item Included features of \textbf{accepting/rejecting booking request} and seeing pending applications from the Verification Side.\\[-1em]
\end{itemize}
\large {\textbf{Query Answering over Linked Data (QALD)}} \normalsize \href{https://github.com/shmundhra/QALD}{\hspace{0.5ex}\faMousePointer}
{\hfill} \textbf{Autumn'18}\\[-1.8em]
\begin{itemize}
    \item Translated \textbf{natural language query into SPARQL query} and \textbf{retrieved answers} to the query from an \textbf{RDF store}.\\[-2em]
    \item Explored various NLP based libraries and frameworks such as Stanford CoreNLP and tried to \textbf{relate semantic information} from the \textbf{generated parse tree} to be able to \textbf{design a SPARQL query to extract answers from DBpedia}.\\[-1em]
\end{itemize}
\iffalse
% \large {tinyC Compiler Design} \normalsize \href{https://github.com/shmundhra/tinyC-Compiler}{\faMousePointer}
% {\hfill} \textbf{Autumn'18}\\[-1.8em]
% \begin{itemize}
%     \item Implemented a compiler based on ISO/IEC 9899:1999 (E) for a subset of C functionalities into x86 Assembly Language.\\[-2em]
%     \item Implemented a Lexical Analyzer using Flex, a Semantic Parser using Bison, and a Machine Independent Code Generator and Translator to convert the Source Code into Three Address Code and finally into x86 Assembly Code.\\[-1em]
% \end{itemize}
\fi
\iffalse
% \large {Single Cycle CPU Design} \normalsize
% \href{https://github.com/shmundhra/KGP-RISC}{\faMousePointer}
% {\hfill} \textbf{Autumn'18}\\[-1.8em]
% \begin{itemize}
%     \item Designed a 32-bit Single Cycle CPU (RISC Architecture) in Xilinx ISE 14.7 using Verilog HDL and burnt on FPGA.\\[-2em]
%     \item Implemented and tested modules like Instruction Fetch, Instruction Decode, ALU individually and then synchronized each module together as one unit.\\[-1em]
% \end{itemize}
\fi
\large {\textbf{Restaurant Automation System (RAS)}} \normalsize \href{https://github.com/shmundhra/Restaurant-Automation-System}
{\hspace{0.5ex}\faMousePointer}
{\hfill} \textbf{Spring'18}\\[-1.8em]
\begin{itemize}
    \item Developed a \textbf{Desktop Application} on Java NetBeans using \textbf{Swing and MySQL} to automate all activities in a restaurant.\\[-2em]
    \item Incorporated features like order placement, inventory management, vendor management, and menu modification.\\[-2em]
    \item Tested the software using \textbf{JUnit Testing technique} with a \textbf{well-rounded test suite} to debug the errors. \\[-2em]
    \item Employed industrial software development techniques including preparing \textbf{SRS}, \textbf{DFD} and \textbf{UML Diagrams}. \\[-2em]
\end{itemize}

\roottitle{\Large{COURSEWORK INFORMATION}}
\spacedhrule{0.1ex}{1.0ex}
\begin{indentsection}
	\skill{Completed with Laboratory Component}
	{Programming and Data Structures, Algorithms I, Introduction to Electronics, Signals and Networks, Software Engineering, Switching Circuits and Logic Design, Computer Organisation and Architecture, Compilers, Operating Systems, Computer Networks, Database Management Systems }
	\skill{Completed}{Mathematics I-II , Discrete Structures, Probability and Statistics, Formal Language and Automata Theory, Algorithms II, Knowledge Modeling and Semantic Technologies, Linear Algebra, Machine Learning}
	\skill{Ongoing}{Advancements in OS Design, Artificial Intelligence,  Natural Language Processing, Object Oriented Systems, Parallel Algorithms, Theory of Computation}
\end{indentsection}

\roottitle{\Large{SKILLS and EXPERTISE}}
\spacedhrule{0.1ex}{1.0ex}
\begin{indentsection}
	\skill{Languages/OS}{C, C++, Java, JSP, Python, UML, MySQL, MIPS, Bazel, Windows, Ubuntu}
	\skill{Tools/Libraries}{C++ STL, NumPy, Pandas, Scikit, NLTK, NetBeans, Swing, Git, Arcanist/Phabrigator, MS Office }
\end{indentsection}

\vspace{-2.0ex}
\roottitle{\Large{AWARDS and ACHIEVEMENTS}}
\spacedhrule{0.1ex}{1.0ex}
\begin{itemize}[leftmargin=*]
\item Holding \textbf{DepartmentRank 4} among the B.Tech students of the Department of Computer Science \& Engineering
{\hfill}\textbf{Aug'19}\\[-2em]
\item \textbf{Peak Rating} \textbf{1977} on CodeChef, \textbf{1726} on Codeforces and \textbf{Level 7} on InterviewBit
\href{https://github.com/shmundhra/Credentials/tree/master/Competitive\%20Programming/Online\%20Judges\%20Profiles} {\hspace{0.5ex}\footnotesize\faMousePointer}
{\hfill}\textbf{Aug'19}\\[-2em]
\item \textbf{Qualified} for Google Code Jam 2019 - Round 2
\href{https://github.com/shmundhra/Credentials/tree/master/Competitive\%20Programming/Google\%20CodeJam} {\hspace{0.5ex}\footnotesize\faMousePointer}
{\hfill}\textbf{May'19}\\[-2em]
\item \textbf{Awarded} the Merit-Cum Means Scholarship by the institute for excellent academic performances in 2018
\href{https://github.com/shmundhra/Credentials/tree/master/Scholarships} {\hspace{0.5ex}\footnotesize\faMousePointer}
{\hfill}\textbf{Mar'19}\\[-2em]
\item \textbf{Acquired} a Rank of 45 in ACM-ICPC Amritapuri-Coimbatore Regionals Onsite Finals
\href{https://github.com/shmundhra/Credentials/tree/master/Competitive\%20Programming/ACM\%20ICPC\%202018} {\hspace{0.5ex}\footnotesize\faMousePointer}
{\hfill}\textbf{Dec'18}\\[-2em]
\item \textbf{Acquired} a Rank of 139 in ACM-ICPC Online Contest and qualified for Amritapuri-Coimbatore Regionals
\href{https://github.com/shmundhra/Credentials/tree/master/Competitive\%20Programming/ACM\%20ICPC\%202018} {\hspace{0.5ex}\footnotesize\faMousePointer}
{\hfill}\textbf{Oct'18}\\[-2em]
\item \textbf{Qualified} for Facebook Hacker Cup 2018 - Round 2 
\href{https://github.com/shmundhra/Credentials/tree/master/Competitive\%20Programming/Facebook\%20Hacker\%20Cup} {\hspace{0.5ex}\footnotesize\faMousePointer}
{\hfill}\textbf{July'18}\\[-2em]
\item \textbf{Awarded} by the Department of Computer Science \& Engineering for performance par excellence in 2017
\href{https://github.com/shmundhra/Credentials/tree/master/Scholarships} {\hspace{0.5ex}\footnotesize\faMousePointer}
{\hfill}\textbf{Apr'18}\\[-2em]
\item \textbf{Awarded} the Batch of 1985 Scholarship by the Institute for excellent academic performances in 2016-17
\href{https://github.com/shmundhra/Credentials/tree/master/Academics} {\hspace{0.5ex}\footnotesize\faMousePointer}
{\hfill}\textbf{Mar'18}\\[-2em]
\item \textbf{Changed Department} to Computer Science \& Engineering by acquiring a institute rank of 9 in the first year
\href{https://github.com/shmundhra/Credentials/tree/master/Academics} {\hspace{0.5ex}\footnotesize\faMousePointer}
{\hfill}\textbf{July'17}\\[-2em]
\item \textbf{Runners-Up Company} at the 11th Asian Regional Space Settlement Design Competition organised \\ in collaboration with Atlantis Research and the Kennedy Space Center
\href{https://github.com/shmundhra/Credentials/tree/master/Academics} {\hspace{0.5ex}\footnotesize\faMousePointer}
{\hfill}\textbf{Jan'15}\\[-2em]
\item \textbf{Scored} a percentile of 98.16 in Qualitative Reasoning, 99.46 in Language Conventions and 98.47 in Quantitative \\Reasoning in Nationwide Problem Solving Assessment Examination (PSA) conducted by CBSE
\href{https://github.com/shmundhra/Credentials/tree/master/Academics} {\hspace{0.5ex}\footnotesize\faMousePointer}
{\hfill}\textbf{Nov'14}\\[-2em]
\end{itemize}

\vspace{-1.3ex}
\roottitle{\Large{POSITIONS of RESPONSIBILITY}}
\spacedhrule{0.1ex}{2.0ex}
\large { \textbf{Tech Lead} at CodeClub, IIT Kharagpur} \normalsize
\href{https://github.com/shmundhra/Credentials/tree/master/Positions\%20of\%20Responsibility} {\hspace{0.5ex}\faMousePointer}
{\hfill} \textbf{Oct'17 - Present}\\[-1.8em]
\begin{itemize}
    %\item Organised and Lectured a \textbf{Competitive Programming Workshop} for the students of IIT Kharagpur.\\[-2em]
    \item Organised an \textbf{HSBC powered Hackathon} in campus for the students of IIT Kharagpur.\\[-2em]
    \item Organised \textbf{up.AI}, a one of a kind flagship event solely dedicated to the use of AI for Social Good.\\[-2em]
    \item Organized \textbf{Code.Fun.Do}, a Microsoft sponsored hackathon which involved the participation from various institutes.\\[-2em]
    %\item Organized workshops, seminars and interactive sessions pertaining to various topics including ML, AI and GitHub.\\[-2em]
    % \item Problem Setter and Tester in \textbf{CodeNites}, an intra KGP coding competition in association with HackerEarth.\\[-2em]
    \item \textbf{Head - Technical Blogs} on Programming at 
    \href{https://medium.com/@codeclub.iitkgp}    {https://medium.com/@codeclub.iitkgp}.
\end{itemize}
\iffalse
% \large { \textbf{Student Mentor} at Student Welfare Group, IIT Kharagpur} \normalsize
% \href{https://github.com/shmundhra/Credentials/tree/master/Positions\%20of\%20Responsibility} {\hspace{0.5ex}\faMousePointer}
% {\hfill} \textbf{Aug'18 - Present}\\[-1.8em]
% \begin{itemize}
%     \item \textbf{Mentor} to 5 students of the junior batch, act as the first stop for all their academic and personal doubts regarding the Institute and the prospects moving forward.\\[-1em]
% \end{itemize}
\fi
\large { \textbf{Volunteer} at National Service Scheme, IIT Kharagpur} \normalsize
{\hfill} \textbf{July'16 - Apr'18}\\[-1.8em]
\begin{itemize}
    \item \textbf{Taught} the students of a primary school for an hour each weekend in a village in the Porapara District of West Bengal. \\[-2em]
    \item \textbf{Conducted Surveys} in a village in the Porapara District of West Bengal to learn about their grievances and act on them. \\[-2em]
    % \item \textbf{Part of Annual NSS Camp} where we built a road in the Raghunathpur District of West Bengal.\\[-1em]
\end{itemize}
\iffalse
% \large { \textbf{Captain} at Student Council, Birla High School} \normalsize
% \href{https://github.com/shmundhra/Credentials/tree/master/Positions\%20of\%20Responsibility} {\hspace{0.5ex}\faMousePointer}
% {\hfill} \textbf{July'14 - Mar'15}\\[-1.8em]
% \begin{itemize}
%     \item \textbf{Elected} by the students as part of the mediating body between the administration and the students.
% \end{itemize}
\fi
\end{document}

