\documentclass[10pt]{article}
\usepackage[a4paper, left=0.4in, right=0.4in, top=0.25in, bottom=0.155in]{geometry}
\usepackage[utf8]{inputenc}
\usepackage[T1]{fontenc}
\usepackage[english]{babel}
\usepackage{fontawesome}
\usepackage[dvipsnames]{xcolor}
\hyphenation{Some-long-word}
\definecolor{darkgray}{rgb}{0.66, 0.66, 0.66}

\usepackage{resume}

\begin{document}

\begin{center}
    {\Huge HIMANSHU \textbf{MUNDHRA}}\\[0.7ex]
\end{center}
% \begin{tabular*}{\textwidth}{@{}l @{\extracolsep{\fill}} r@{}}
%   \begin{tabular}{@{}l@{}} 
%      AE-312, MS Hall of Residence,\\ 
%      \href{http://www.iitkgp.ac.in}{Indian Institute Of Technology, Kharagpur}\\
%      West Bengal, India - 721302
%   \end{tabular} & 
%   \begin{tabular}{@{}r@{}} 
%     \faMobile\hspace{0.2ex} \textbf{+91 91639 95974} \\  
%     \Letter\hspace{0.2ex}
%     \href{mailto:himanshumundhra98@gmail.com}{himanshumundhra98@gmail.com} \\
%     \faLinkedinSquare\hspace{0.2ex}  \href{https://www.linkedin.com/in/shmundhra/}
%     {\tt linkedin.com/in/shmundhra}  \\
%     \faGithub\hspace{0.2ex} 
%     \href{https://github.com/shmundhra/}{\tt github.com/shmundhra}  \\
%     % \faMedium\hspace{0.2ex} \href{https://medium.com/@himanshumundhra98}
%     % {\tt medium.com/@himanshumundhra98}  \\
%   \end{tabular}
% \end{tabular*}
\vspace{-3.5ex}
\begin{center}
    {\small AE-312, MS Hall of Residence, IIT Kharagpur, West Bengal, India - 721 302 | \faMobile\hspace{0.2ex} \textbf{+91 91639 95974} }
\end{center}
\vspace{-3.8ex}
\begin{center}
    \Letter\hspace{0.2ex}
    \href{mailto:himanshumundhra98@gmail.com}{\small himanshumundhra98@gmail.com} |
    \Letter\hspace{0.2ex}
    \href{mailto:himanshu.mundhra@iitkgp.ac.in}{\small himanshu.mundhra@iitkgp.ac.in} | 
    \faLinkedinSquare\hspace{0.2ex}
    \href{https://www.linkedin.com/in/shmundhra/}{\small shmundhra} | 
    \faGithub\hspace{0.2ex}
    \href{https://github.com/shmundhra/}{\small shmundhra} | 
    \faMedium\hspace{0.2ex}
    \href{https://medium.com/@himanshumundhra98}{\small himanshumundhra98}
\end{center}

\vspace{-3.0ex}
\roottitle{\Large{EDUCATION}}
\spacedhrule{0.2ex}{2.0ex}
\vspace{-1ex}
\renewcommand{\arraystretch}{1.5}
\indent \begin{tabular}{ |@{\hskip 0.125in}l @{\hskip 0.125in} |@{\hskip 0.125in}l @{\hskip 0.125in} |@{\hskip 0.125in}l @{\hskip 0.125in} |@{\hskip 0.125in}l @{\hskip 0.125in} |l }
\hline \textbf{Degree} & \textbf{Institute / Board} & \textbf{Year} & \textbf{CGPA / \%} \\ 
\hline {B. Tech} in Computer Science and Engineering & IIT Kharagpur  & 2016 - 2020 (Expected) & \textbf{9.57 / 10} \href{https://github.com/shmundhra/Credentials/tree/master/Academics} {\hspace{1.0ex}\faMousePointer} \\
\hline All India {Senior School} Certificate Examination & Birla High School - CBSE & March, 2016 & \textbf{95.2 \%} \href{https://github.com/shmundhra/Credentials/tree/master/Academics} {\hspace{2.5ex}\faMousePointer}\\
\hline All India {Secondary School} Examination & Birla High School - CBSE & March, 2014 & \textbf{10 / 10 } \href{https://github.com/shmundhra/Credentials/tree/master/Academics} {\hspace{2.3ex}\faMousePointer}\\
\hline
\end{tabular}
\\

\vspace{-1.5ex}
\roottitle{\Large{INTERNSHIP}}
\spacedhrule{0.15ex}{1.0ex}
\large {\textbf{Member of Technical Staff Intern at Rubrik, Inc.}} \normalsize
\href{https://github.com/shmundhra/Credentials/tree/master/Internships} {\hspace{0.5ex}\faMousePointer}
{\hfill} \textbf{Summer'19}\\[0.1em]%[-1.75em]
The current methods for large data transfer in files or streams were computationally expensive on the host side and slow. Moreover, the data transfer in some products was taking place in a serialised manner, leading to low throughput, especially on the high latency replication links.\\
My task was to \textbf{build from scratch} a \textbf{High-Throughput Pipeline-able Data Streaming Library} with a minimal overhead above the TCP to support smooth and fast data flow between two hosts. This library provided support to both \textbf{rewindable and non-rewindable} producer and consumers on a \textbf{secured duplex channel}.
\iffalse
\begin{itemize}
\item Designed a High Throughput Streaming Library for data transfer over a \textbf{Secured TCP Connection} as an API to user.\\[-1.9em]
\item Implemented a \textbf{duplex server-client stream} where each end can be a rewindable/non-rewindable   producer/consumer.\\[-1.9em]
\item \textbf{Designed} the response, request and error packet \textbf{headers} and the \textbf{protocol} to be followed from scratch.\\[-1.9em]
%\item Implemented \textbf{multiple classes in a hierarchical structure} and developed different utilities and wrapper modules.\\[-1.9em]
\item Used \textbf{thread-per-connection concurrency control} mechanism and maintained synchronisation between threads using \textbf{mutex lock wrappers} to avoid the producer-consumer and stream registration race conditions.\\[-2em]
\end{itemize}
\fi
\vspace{-0.5ex}
\roottitle{\Large{TEACHING EXPERIENCE}}
\spacedhrule{0.15ex}{1.0ex}
\large { \textbf{Teaching Assistant} for Algorithms-I CS20003, IIT Kharagpur} \normalsize
\href{https://github.com/shmundhra/Credentials/tree/master/Teaching\%20Ventures} {\hspace{0.5ex}\faMousePointer}
{\hfill} \textbf{Jul'19 - Present}\\[-1.9em]
\begin{itemize}
    \item Organise tutorials for the students, set practice problems and solve them in the class of 120 along with doubt clearing.\\[-1.25em]
\end{itemize}
\large { \textbf{Lecturer} at Competitive Programming Workshop, IIT Kharagpur} \normalsize
\href{https://github.com/shmundhra/Credentials/tree/master/Teaching\%20Ventures} {\hspace{0.5ex}\faMousePointer}
{\hfill} \textbf{Jan'19 - Apr'19}\\[-1.9em]
\begin{itemize}
    \item Designed and Lectured an Intermediate Competitive Programming Workshop for the students of IIT Kharagpur.\\[-2em]
\end{itemize}

\vspace{-0.5ex}
\roottitle{\Large{MAJOR PROJECTS}}
\spacedhrule{0.15ex}{1.0ex}
\large {\textbf{Live Modifiable Server}} \normalsize \href{}{} {\hfill} \textbf{Ongoing}\\[-1.75em]
\begin{itemize}
    \item Aim to implement a \textbf{live modifiable server}, where changes in code are immediately reflected in the running executable.\\[-1.9em]
    \item A WrapperServer Program ensures that while the modified source code is being restarted in a separate thread, the previous \textbf{state of the connection is not lost} and the program begins execution from the paused state.\\[-1em]
\end{itemize}
\large {\textbf{Multi Target Stance Detection using Graph Convolution Networks}} \normalsize \href{}{} {\hfill} \textbf{Ongoing}\\[-1.75em]
\begin{itemize}
    \item Aim to \textbf{assign a stance to textual data} by a user catering to a particular target or a set of related targets using GCNs.\\[-1.9em]
    \item \textbf{TextGCNs} generate a \textbf{multi-layer graph} that will incorporate the user history and comments on the topic of interest.\\[-1.9em]
    \item Aim to utilise \textbf{user background information} to be able to predict their stance in a more accurate manner.\\[-1.9em]
    \item GCNs {Semi-supervised framework} allows us to train a small dataset and achieve results similar to full-supervision.\\[-1em]
\end{itemize}
\large {\textbf{Loadable Kernel Module}} \normalsize \href{https://github.com/shmundhra/Systems-Programming/blob/master/LKM/LKM_BST.pdf} {\hspace{0.5ex}\faGithub} {\hfill} \textbf{Autumn'19}\\[-1.75em]
\begin{itemize}
    \item Created a world-readable and writable user-space interface to the LKM as a file in the \textbf{/path/proc} folder. \\[-1.9em]
    \item The LKM \textbf{stores data in a BST} and reads data node by node in each read call in user determined order of tree traversal. \\[-1.9em]
    \item The LKM \textbf{handles concurrency and mutual exclusion of data} from \textbf{multiple user-space programs}.\\[-1em]
\end{itemize}
\large {\textbf{Memory-Resident Unix-Like File System}} \normalsize  \href{https://github.com/shmundhra/Systems-Programming/tree/master/File_System} {\hspace{0.5ex}\faGithub} {\hfill} \textbf{Spring'19}\\[-1.75em]
\begin{itemize}
    \item Created a \textbf{Multi-Level Directory Tree like File System} which supports all Linux-type file commands.\\[-1.9em]
    \item \textbf{Linked List Implementation} where the {Free Blocks are a Bit Vector} and {Data Blocks are maintained in a FAT}.\\[-1.9em]
    \item \textbf{iNode Implementation} where the {Free Blocks are a Linked List}  and the {File Blocks are maintained in iNodes}.\\[-1em]
\end{itemize}

\vspace{-2ex}
\roottitle{\Large{AWARDS and ACHIEVEMENTS}}
\spacedhrule{0.15ex}{1.0ex}
\begin{itemize}[leftmargin=*]
\item Holding \textbf{InstituteRank 10} among the B.Tech students of the Indian Institute of Technology, Kharagpur {\hfill}\textbf{Sep'19}\\[-1.8em]
\item Holding \textbf{DepartmentRank 4} among the B.Tech students of the Department of Computer Science \& Engineering {\hfill}\textbf{Sep'19}\\[-1.8em]
\item \textbf{Peak Rating} \textbf{1977} on CodeChef, \textbf{1726} on Codeforces and \textbf{Level 7} on InterviewBit \href{https://github.com/shmundhra/Credentials/tree/master/Competitive\%20Programming/Online\%20Judges\%20Profiles} {\hspace{0.5ex}\footnotesize\faMousePointer} {\hfill}\textbf{Sep'19}\\[-1.8em]
\item Qualified for \textbf{Google Code Jam - Round 2} and \textbf{Facebook Hacker Cup - Round 2} \href{https://github.com/shmundhra/Credentials/tree/master/Competitive\%20Programming/Google\%20CodeJam} {\hspace{1ex}\footnotesize\faMousePointer} \href{https://github.com/shmundhra/Credentials/tree/master/Competitive\%20Programming/Facebook\%20Hacker\%20Cup} {\hspace{0.5ex}\footnotesize\faMousePointer} {\hfill}\textbf{May'19}\\[-1.8em]
\item Acquired a \textbf{Rank of 45} in ACM-ICPC Amritapuri-Coimbatore Regionals Onsite Finals \href{https://github.com/shmundhra/Credentials/tree/master/Competitive\%20Programming/ACM\%20ICPC\%202018} {\hspace{0.5ex}\footnotesize\faMousePointer} {\hfill}\textbf{Dec'18}\\[-1.8em]
%\item Acquired a \textbf{Rank of 139} in ACM-ICPC Online Contest and qualified for Amritapuri-Coimbatore Regionals \href{https://github.com/shmundhra/Credentials/tree/master/Competitive\%20Programming/ACM\%20ICPC\%202018} {\hspace{0.5ex}\footnotesize\faMousePointer} {\hfill}\textbf{Oct'18}\\[-1.8em]
\item \textbf{Awarded} by the Department of Computer Science \& Engineering for performance par excellence in 2017 \href{https://github.com/shmundhra/Credentials/tree/master/Scholarships} {\hspace{0.5ex}\footnotesize\faMousePointer} {\hfill}\textbf{Apr'18}\\[-1.8em]
\item \textbf{Awarded} the Batch of 1985 Scholarship by the Institute for excellent academic performances in 2016-17 \href{https://github.com/shmundhra/Credentials/tree/master/Academics} {\hspace{0.5ex}\footnotesize\faMousePointer} {\hfill}\textbf{Mar'18}\\[-1.8em]
\item \textbf{Changed Department} to Computer Science \& Engineering by acquiring an \textbf{InstituteRank 9} in the first year \href{https://github.com/shmundhra/Credentials/tree/master/Academics} {\hspace{0.5ex}\footnotesize\faMousePointer} {\hfill}\textbf{Jul'17}\\[-1.8em]
\item Acquired a \textbf{top 1.22\%} rank in JEE Advanced-2016 and \textbf{top 0.32\%} rank in JEE Mains-2016. \href{https://github.com/shmundhra/Credentials/tree/master/Academics} {\hspace{0.5ex}\footnotesize\faMousePointer} {\hfill}\textbf{Jun'16}\\[-1.8em]
%\item \textbf{Awarded} the Merit-Cum Means Scholarship by the institute for excellent academic performances in 2018 \href{https://github.com/shmundhra/Credentials/tree/master/Scholarships} {\hspace{0.5ex}\footnotesize\faMousePointer} {\hfill}\textbf{Mar'19}\\[-1.8em]
%\item Qualified for \textbf{Facebook Hacker Cup - Round 2} \href{https://github.com/shmundhra/Credentials/tree/master/Competitive\%20Programming/Facebook\%20Hacker\%20Cup} {\hspace{0.5ex}\footnotesize\faMousePointer} {\hfill}\textbf{July'18}\\[-1.8em]
%\item \textbf{Runners-Up Company} at the 11th Asian Regional Space Settlement Design Competition organised \\ in collaboration with Atlantis Research and the Kennedy Space Center \href{https://github.com/shmundhra/Credentials/tree/master/Academics} {\hspace{0.5ex}\footnotesize\faMousePointer} {\hfill}\textbf{Jan'15}\\[-1.8em]
%\item \textbf{Scored} a percentile of 98.16 in Qualitative Reasoning, 99.46 in Language Conventions and 98.47 in Quantitative \\Reasoning in Nationwide Problem Solving Assessment Examination (PSA) conducted by CBSE \href{https://github.com/shmundhra/Credentials/tree/master/Academics} {\hspace{0.5ex}\footnotesize\faMousePointer} {\hfill}\textbf{Nov'14}\\[-1.8em]
\end{itemize}

\roottitle{\Large{TERM PROJECTS}}
\spacedhrule{0.15ex}{1.0ex}
\large {\textbf{Virtual Memory using Demand Paging}} \normalsize  \href{https://github.com/shmundhra/Systems-Programming/tree/master/Demand_Paging} {\hspace{0.5ex}\faGithub} {\hfill} \textbf{Spring'19}\\[-1.75em]
\begin{itemize}
    \item Created \textbf{different modules} such as Master, Scheduler, Processes and the Memory Management Unit (MMU).\\[-1.9em]
    \item Implemented message passing between modules through     \textbf{Blocking Synchronous IPC Message Queues}.\\[-1.9em]
    \item Accessed \textbf{Shared Memory} synchronously using \textbf{signals \& messages} to indicate safe and mutually exclusive access.
    % \\[-1.9em] \item \textbf{Proportional allocation} of frames along with an \textbf{LRU strategy} for the Page Table and the TLB was adopted.
    \\[-1em]
\end{itemize}
\vspace{-0.5ex}
\large {\textbf{Reliable User Datagram Protocol}} \normalsize  \href{https://github.com/shmundhra/Socket-Programming/tree/master/My\%20Reliable\%20UDP} {\hspace{0.5ex}\faGithub} {\hfill} \textbf{Spring'19}\\[-1.75em]
\begin{itemize}
    \item Created a \textbf{Static Library} with all required functions for our protocol - socket(), send(), recv(), close().\\[-1.9em]
    \item Created a \textbf{Concurrent Thread} which managed the Receiving of Messages and placed them into the Receive Buffer.\\[-1.9em]
    \item This Thread also managed the \textbf{Acknowledgements and the Re-transmissions} to ensure reliability.\\[-1em]
\end{itemize}
\vspace{-0.5ex}
\large {\textbf{Auditorium and Room Booking System (HOVA)}} \normalsize  \href{https://github.com/shmundhra/HOVA} {\hspace{0.5ex}\faGithub} {\hfill} \textbf{Spring'19}\\[-1.75em]
\begin{itemize}
    \item Developed a \textbf{Web Application} on Java NetBeans using \textbf{JSP and MySQL} to automate room booking in IIT Kharagpur.\\[-1.9em]
    %\item Incorporated the \textbf{hierarchical structure} of room booking by including verification from the Department, Authority, Security and the AV Cell, in that order.\\[-1.9em]
    \item Included dynamic features like submitting/accepting booking request at both the Applicant and Verification Side.\\[-1em]
    % \item Included features of \textbf{submitting/cancelling booking requests} and seeing pending applications from the Applicant Side.\\[-1.9em]
    % \item Included features of \textbf{accepting/rejecting booking request} and seeing pending applications from the Verification Side.\\[-1em]
\end{itemize}
\vspace{-0.5ex}
\large {\textbf{Query Answering over Linked Data (QALD)}} \normalsize \href{https://github.com/shmundhra/QALD}{\hspace{0.5ex}\faGithub} {\hfill} \textbf{Autumn'18}\\[-1.75em]
\begin{itemize}
    \item Translated \textbf{natural language query into SPARQL query} and \textbf{retrieved answers} to the query from an \textbf{RDF store}.\\[-1.9em]
    \item Explored various NLP based libraries and frameworks such as Stanford CoreNLP and tried to \textbf{relate semantic information} from the \textbf{generated parse tree} to be able to \textbf{design a SPARQL query to extract answers from DBpedia}.\\[-1em]
\end{itemize}
% \large {tinyC Compiler Design} \normalsize \href{https://github.com/shmundhra/tinyC-Compiler}{\faGithub} {\hfill} \textbf{Autumn'18}\\[-1.8em]
% \begin{itemize}
%     \item Implemented a compiler based on ISO/IEC 9899:1999 (E) for a subset of C functionalities into x86 Assembly Language.\\[-2em]
%     \item Implemented a Lexical Analyzer using Flex, a Semantic Parser using Bison, and a Machine Independent Code Generator and Translator to convert the Source Code into Three Address Code and finally into x86 Assembly Code.\\[-1em]
% \end{itemize}
% \large {Single Cycle CPU Design} \normalsize \href{https://github.com/shmundhra/KGP-RISC}{\faGithub} {\hfill} \textbf{Autumn'18}\\[-1.8em]
% \begin{itemize}
%     \item Designed a 32-bit Single Cycle CPU (RISC Architecture) in Xilinx ISE 14.7 using Verilog HDL and burnt on FPGA.\\[-2em]
%     \item Implemented and tested modules like Instruction Fetch, Instruction Decode, ALU individually and then synchronized each module together as one unit.\\[-1em]
% \end{itemize}
\vspace{-0.5ex}
\large {\textbf{Restaurant Automation System (RAS)}} \normalsize \href{https://github.com/shmundhra/Restaurant-Automation-System} {\hspace{0.5ex}\faGithub} {\hfill} \textbf{Spring'18}\\[-1.75em]
\begin{itemize}
    \item Developed a \textbf{Desktop Application} on Java NetBeans using \textbf{Swing and MySQL} to automate all activities in a restaurant.\\[-1.9em]
    %\item Incorporated features like order placement, inventory management, vendor management, and menu modification.\\[-1.95em]
    \item Tested the software using \textbf{JUnit Testing technique} with a \textbf{well-rounded test suite} to debug the errors.  \\[-1.9em] \item Employed industrial software development techniques including preparing \textbf{SRS}, \textbf{DFD} and \textbf{UML Diagrams}. 
    \\[-1em]
\end{itemize}
\vspace{-0.5ex}
\large {\textbf{Systems Programming}} \normalsize \href{https://github.com/shmundhra/Systems-Programming}{\faGithub} {\hfill}\textbf{Spring'19 - Ongoing}\\[-1.8em]
\begin{itemize}
    \item Implemented a rudimentary \textbf{Command-Line Interpreter} for Linux on C++ using \textbf{forks and pipes}. \\[-1.9em]
    % \item Developed a \textbf{Process/Thread Scheduler} which implements the common Process Scheduling Algorithms.\\[-1.8em]
    \item Simulated a \textbf{Multi-threaded mutually exclusive deadlock free} Producer-Consumer problem implementation.\\[-1.9em]
    \item Implemented a \textbf{multi-threaded} Sparse-Matrix Multiplication program and analysed change in execution time with number of threads, chunks size assigned to each thread and scheduling algorithms.\\[-1em]
\end{itemize}
\vspace{-0.5ex}    
\large {\textbf{Socket Programming}} \normalsize \href{https://github.com/shmundhra/Socket-Programming}{\faGithub} {\hfill}\textbf{Spring'19 - Ongoing}\\[-1.7em]
\begin{itemize}
    \item Developed an \textbf{iterative FTP Server and FTP Client} following a subset of the File Transfer Protocols.\\[-1.9em] 
    \item Developed a simplistic implementation of a \textbf{Peer-to-Peer Live Chat Relay Server}.\\[-1.9em]
    \item Developed a version of the Linux-Command \textbf{\$traceroute} using \textbf{Raw Sockets} and the \textbf{TTL Field} in the IP Header.
\end{itemize}

\vspace{-2.2ex}
\roottitle{\Large{OTHER PROJECTS}}
\spacedhrule{0.15ex}{1.0ex}
\begin{indentsection}
    \smaller\href{https://github.com/shmundhra?tab=repositories}     {\faGithub}\hspace{0.5ex}\normalsize \textbf{Web Crawlers} {\hfill}{- Developed workable web crawlers for CodeChef, InterviewBit and CreateDebate using BS4 and Requests}
    \\[0.05em] \smaller\href{https://github.com/shmundhra/tinyC-Compiler}     {\faGithub}\hspace{0.5ex}\normalsize \textbf{tinyC Compiler} {\hfill}{- Wrote a Lexical Analyser in Flex, Semantic Parser in Bison and Machine Independent Code Generator}
    %\\[0.05em] \smaller\href{https://github.com/shmundhra/KGP-RISC} {\faMousePointer}\normalsize \textbf{Single Cycle CPU} {\hfill}{- Designed a 32-bit Single Cycle CPU in Verilog with ALU, Instruction Fetch and Decode modules}
    \\[0.05em] \smaller\href{https://github.com/shmundhra/Machine-Learning} {\faGithub}\hspace{0.5ex}\normalsize \textbf{Machine Learning} {\hfill}{- Developed a Regression Tool, a Decision Tree Classifier and a Hierarchical Clustering Tool}
    \\[0.05em] \smaller\href{https://github.com/shmundhra/Natural-Language-Processing} {\faGithub}\hspace{0.5ex}\normalsize \textbf{Natural Language Processing} {\hfill}{- Implemented N-gram models, POS Tagging and Transition based Dependency Parsing} 
    % \smaller\href{https://github.com/shmundhra/Credentials/tree/master/Reports} {\faMousePointer} \normalsize \textbf{Mood Induction using Visuals}     {\hfill}{- Analysed the effect of visuals on the mood of students through experiments in campus}\\[0.05em]
    % \smaller\href{https://github.com/shmundhra/Credentials/tree/master/Reports} {\faMousePointer} \normalsize \textbf{PsyCap Index} {\hfill}{- Analysed the levels of hope, self-efficacy, resilience and optimism in students through survey in campus}
\end{indentsection}

\vspace{-0.6ex}
\roottitle{\Large{COURSEWORK INFORMATION} }%\hspace{0.1ex} \large \href{https://github.com/shmundhra/Credentials/blob/master/Academics/AcademicTranscript_IITKGP.pdf}{\faMousePointer}}
\spacedhrule{0.15ex}{1.0ex}
\begin{indentsection}
	\skill{Completed with Laboratory Component}
	{Algorithms I, Software Engineering, Switching Circuits and Logic Design, Computer Organisation and Architecture, Compilers, Operating Systems, Computer Networks, Database Management Systems }
	\skill{Completed}{Discrete Structures, Probability and Statistics, Formal Language and Automata Theory, Algorithms II, Knowledge Modelling and Semantic Technologies, Linear Algebra, Machine Learning, Advancements in OS Design, Artificial Intelligence,  Natural Language Processing, Object Oriented Systems, Parallel Algorithms, Theory of Computation}
\end{indentsection}

\vspace{-0.5ex}
\roottitle{\Large{SKILLS and EXPERTISE}}
\spacedhrule{0.15ex}{1.0ex}
\begingroup
    \fontsize{10pt}{12pt}\selectfont
    \vspace{-0.7ex}
	\skill{Languages/ OS }{\hfill}
	{C,\hspace{0.75ex} C++,\hspace{0.75ex} Python,\hspace{0.75ex} UML,\hspace{0.75ex} MySQL,\hspace{0.75ex} Java,\hspace{0.75ex} JSP,\hspace{0.75ex} LaTeX,\hspace{0.75ex} MIPS,\hspace{0.75ex} Windows,\hspace{0.75ex} Ubuntu}
	\vspace{-0.7ex}
	\skill{Tools }{\hfill}
	{Git,\hspace{0.75ex} Netbeans,\hspace{0.75ex} Swing,\hspace{0.75ex} VSCode,\hspace{0.75ex} Sublime Text,\hspace{0.75ex} Arcanist/Phabrigator,\hspace{0.75ex} Vim}
	\vspace{-0.7ex}
    \skill{Libraries }{\hfill}
    {C++ STL,\hspace{0.75ex} NumPy,\hspace{0.75ex} Pandas,\hspace{0.75ex} Matplotlib,\hspace{0.75ex} Scikit,\hspace{0.75ex} NLTK,\hspace{0.75ex} OpenMP,\hspace{0.75ex}  BeautifulSoup,\hspace{0.75ex} Requests }
\endgroup

\vspace{-0.3ex}
\roottitle{\Large{POSITIONS of RESPONSIBILITY}}
\spacedhrule{0.15ex}{2.0ex}
\vspace{-1ex}
\large { \textbf{Tech Lead} at CodeClub, IIT Kharagpur} \normalsize
\href{https://github.com/shmundhra/Credentials/tree/master/Positions\%20of\%20Responsibility} {\hspace{0.5ex}\faMousePointer}
{\hfill} \textbf{Oct'17 - Present}\\[-1.8em]
\begin{itemize}
    %\item Organised and Lectured a \textbf{Competitive Programming Workshop} for the students of IIT Kharagpur.\\[-2em]
    \item Organised an \textbf{HSBC powered Hackathon} in campus for the students of IIT Kharagpur.\\[-2em]
    \item Organised \textbf{up.AI}, a one of a kind flagship event solely dedicated to the use of AI for Social Good.\\[-2em]
    \item Organized \textbf{Code.Fun.Do}, a Microsoft sponsored hackathon which involved the participation from various institutes.\\[-2em]
    %\item Organized workshops, seminars and interactive sessions pertaining to various topics including ML, AI and GitHub.\\[-2em]
    % \item Problem Setter and Tester in \textbf{CodeNites}, an intra KGP coding competition in association with HackerEarth.\\[-2em]
    \item \textbf{Head - Technical Blogs} on Programming at 
    \href{https://medium.com/@codeclub.iitkgp}    {https://medium.com/@codeclub.iitkgp}.
\end{itemize}
\large { \textbf{Student Mentor} at Student Welfare Group, IIT Kharagpur} \normalsize
\href{https://github.com/shmundhra/Credentials/tree/master/Positions\%20of\%20Responsibility} {\hspace{0.5ex}\faMousePointer}
{\hfill} \textbf{Aug'18 - Present}\\[-1.8em]
\begin{itemize}
    \item \textbf{Mentor} to 5 students of the junior batch, act as the first stop for all their academic and personal doubts % regarding the Institute and the prospects moving forward
    .
\end{itemize}
% \large { \textbf{Volunteer} at National Service Scheme, IIT Kharagpur} \normalsize
% {\hfill} \textbf{July'16 - Apr'18}\\[-1.8em]
% \begin{itemize}
%     \item \textbf{Taught} the students of a primary school for an hour each weekend in a village in the Porapara District of West Bengal. \\[-2em]
%     \item \textbf{Conducted Surveys} in a village in the Porapara District of West Bengal to learn about their grievances and act on them. \\[-1em]
%     % \item \textbf{Part of Annual NSS Camp} where we built a road in the Raghunathpur District of West Bengal.\\[-1em]
% \end{itemize}
%\large { \textbf{Captain} at Student Council, Birla High School} \normalsize \href{https://github.com/shmundhra/Credentials/tree/master/Positions\%20of\%20Responsibility} {\hspace{0.5ex}\faMousePointer} {\hfill} \textbf{July'14 - Mar'15}\\[-1.8em]
%\begin{itemize}
    %\item \textbf{Elected} by the students as part of the mediating body between the administration and the students.
%\end{itemize}
\end{document}
